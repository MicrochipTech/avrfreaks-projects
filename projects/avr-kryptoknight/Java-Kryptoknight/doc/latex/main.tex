\hypertarget{main_intro_sec}{}\section{Introduction}\label{main_intro_sec}
\hypertarget{main_what_in}{}\subsection{What's in this package}\label{main_what_in}
In this package you can find the Java implementation of the IBM-Kryptoknight protocol. IBM developed this protocol for authentication especially for devices with limited calculation abilities such as small microcontrollers. Here the implementation is made in Java. This will make it possible to authenticate several PC to eachother or to authenticate a microcontroller to a PC. In contrast with the AVR-Kryptoknight package for the Atmel ATMEGA8535 microcontroller, this package can function as a station waiting for an authentication request. This is the so called BOB-mode. The other available mode is the ALICE-mode. Alice always starts the authentication procedure.\hypertarget{main_References}{}\section{References}\label{main_References}
There is a lot information available on the internet about authentication. I focused on the Kryptoknight-protocol because it's a lightweight protocol that suited my needs best.\hypertarget{main_Authentication}{}\subsection{Authentication}\label{main_Authentication}
\begin{itemize}
\item BIRD R., et al., Systematic Design of a Family of Attack-Resistant Authentication Protocols \item BIRD R., et al., The \hyperlink{class_kryptoknight}{Kryptoknight} Family of Light-weight Protocols for Authentication and Key Distribution, December 1993 \item CLARK J., JACOB J. A Survey of Authentication Protocol Literature: Version 1.0 \item COMPTECHDOC, Authentication protocols \href{http://www.comptechdoc.org/}{\tt http://www.comptechdoc.org/} independent/networking/protocol/protauthen.html, December 3, 2004 \item CRISPO B., et al., Symmetric Key Authentication Services Revisited \item INFOSYSSEC, The Security Portal for Information System Security Professionals \href{http://www.infosyssec.com/infosyssec/secauthen1.htm}{\tt http://www.infosyssec.com/infosyssec/secauthen1.htm} \item JANSON P., et al. Scalability and Flexibility in Authentication Services: The \hyperlink{class_kryptoknight}{Kryptoknight} approach, 1997 \item MOLVA K., et al., \hyperlink{class_kryptoknight}{Kryptoknight} Authentication and Key Distribution System \item MOLVA K., et al., Authentication of Mobile Users, August 20, 1993 \item RFC1994 - PPP Challenge Handshake Authentication Protocol (CHAP) \item TSUDIK G. VAN HERREWEGHEN E., On Simple and Secure Key Distribution\end{itemize}
\hypertarget{main_Hmac}{}\subsection{Hmac}\label{main_Hmac}
\begin{itemize}
\item BELLARE M., et al., Keying Hash Functions for Message Authentication \item FIPS PUB 198 The Keyed-Hash Message Authentication Code (HMAC) \item IETF, RFC2104 HMAC: Keyed-Hashing for Message Authentication, February 1997\end{itemize}
\hypertarget{main_Hashing}{}\subsection{Hashing}\label{main_Hashing}
\begin{itemize}
\item FIPS PUB 180-2 Announcing the Secure Hash Standard, August 1, 2002 \item HUSNI A. Implementation Secure Hash Standard with AVR Microcontroller \item IETF, RFC3174 US Secure Hash Algorithm 1 (SHA1)\end{itemize}
\hypertarget{main_secRand}{}\subsection{Random Number Generators}\label{main_secRand}
\begin{itemize}
\item BERNSTEIN G.M., et al., Method and Apparatus for generating Secure Random Numbers using Chaos, US Patent US5007087, April 9, 1991. \item DAVIES R., Hardware Random Number Generators, \href{http://www.robertnz.net/hwrng.htm}{\tt http://www.robertnz.net/hwrng.htm} \item GILLEY J.E., Method and Apparatus for generating truly random numbers, US Patent US5781458, July 14, 1998 \item GOLBECK E. C., Random Bit Stream Generator and Method, US Patent US5239494, August 24, 1993. \item HOFFMAN E.J., Random Number Generator, US Patent US6061702, May 9 2000 \item INTEL, \href{http://web.archive.org/web/20040229005150/http://developer.intel.com/design/security/rng/rngppr.htm}{\tt http://web.archive.org/web/20040229005150/http://developer.intel.com/design/security/rng/rngppr.htm} \item JAKOBSSON M., et al. A Physical Secure Random Bit Generator \item JUN B.,et al., The Intel Random Number Generator, April 22, 1999 \item LOGUE A., Hardware Random Number Generator, May 2002 \href{http://www.cryogenius.com/hardware/rng/}{\tt http://www.cryogenius.com/hardware/rng/} \item MESSINA M. et al., Random Bit Sequence Generator, US Patent US2003/0093455 A1, May 15, 2003 \item RSA, Hardware-Based Random Number Generation, An RSA Data Security White Paper \item SKLAVOS N., Random Number Generator Architecture And Vlsi Implementation, May 2002 \item UNER E., Generating Random Numbers, Embedded.com \href{http://www.embedded.com/showArticle.jhtml?articleID=20900500}{\tt http://www.embedded.com/show\-Article.jhtml?article\-ID=20900500} \item WELLS S.E., Programmable Random Bit Source, US Patent US6795837, September 21, 2004 \item WELLS S.E., et al.Secure Handware Random Number Generator, US Patent US6792438, 14 sept 2004 \item WIKIPEDIA, Hardware Random Number Generator, \href{http://en.wikipedia.org/wiki/Hardware_random_number_generator}{\tt http://en.wikipedia.org/wiki/Hardware\_\-random\_\-number\_\-generator} \item WILLWARE, Hardware Random Bit Generator, \href{http://willware.net:8080/hw-rng.html}{\tt http://willware.net:8080/hw-rng.html}\end{itemize}
\hypertarget{main_secNoise}{}\subsection{Random Noise Sources}\label{main_secNoise}
\begin{itemize}
\item ELECTRONIC DESIGN, Wide-Band Analog White-Noise Generator, November 3, 1997, \href{http://www.elecdesign.com/Articles/Index.cfm?AD=1&ArticleID=6356}{\tt http://www.elecdesign.com/Articles/Index.cfm?AD=1\&Article\-ID=6356} \item SHUPE C.D., Random Voltage Source With Subtstantially Uniform Distribution, US Patent US4578649 \end{itemize}
