\hypertarget{_c_r_c___test_8c}{
\section{CRC\_\-Test.c File Reference}
\label{_c_r_c___test_8c}\index{CRC_Test.c@{CRC\_\-Test.c}}
}
{\tt \#include \char`\"{}define.h\char`\"{}}\par
\subsection*{Functions}
\begin{CompactItemize}
\item 
uint16\_\-t \hyperlink{_c_r_c___test_8c_a0}{get\_\-crc} (uint8\_\-t $\ast$buf, uint8\_\-t length)
\item 
\hyperlink{define_8h_a29}{BOOL} \hyperlink{_c_r_c___test_8c_a1}{strip\-CRC} (\hyperlink{struct_f_r_a_m_e_r}{FRAMER} $\ast$p\-Framer)
\item 
void \hyperlink{_c_r_c___test_8c_a2}{append\-CRC} (uint8\_\-t $\ast$buffer, int length)
\end{CompactItemize}


\subsection{Detailed Description}
The protocol makes use of checksums, these functions implement these. 

\subsection{Function Documentation}
\hypertarget{_c_r_c___test_8c_a2}{
\index{CRC_Test.c@{CRC\_\-Test.c}!appendCRC@{appendCRC}}
\index{appendCRC@{appendCRC}!CRC_Test.c@{CRC\_\-Test.c}}
\subsubsection[appendCRC]{\setlength{\rightskip}{0pt plus 5cm}void append\-CRC (uint8\_\-t $\ast$ {\em buffer}, int {\em length})}}
\label{_c_r_c___test_8c_a2}


Calculate a CRC16 of a buffer and append it to that buffer. \begin{Desc}
\item[Parameters:]
\begin{description}
\item[{\em buffer}]databuffer of which CRC16 will be calculated and to which CRC16 will be appended. \item[{\em length}]number of bytes in the buffer \end{description}
\end{Desc}
\begin{Desc}
\item[Returns:]the same buffer, with CRC-appended (little endian) \end{Desc}
\hypertarget{_c_r_c___test_8c_a0}{
\index{CRC_Test.c@{CRC\_\-Test.c}!get_crc@{get\_\-crc}}
\index{get_crc@{get\_\-crc}!CRC_Test.c@{CRC\_\-Test.c}}
\subsubsection[get\_\-crc]{\setlength{\rightskip}{0pt plus 5cm}uint16\_\-t get\_\-crc (uint8\_\-t $\ast$ {\em buf}, uint8\_\-t {\em length})}}
\label{_c_r_c___test_8c_a0}


Get the CRC16 of a databuffer. \begin{Desc}
\item[Parameters:]
\begin{description}
\item[{\em buf}]containing data of which the CRC will be calculated \item[{\em length}]total length of the buffer \end{description}
\end{Desc}
\begin{Desc}
\item[Returns:]CRC16 \end{Desc}
\hypertarget{_c_r_c___test_8c_a1}{
\index{CRC_Test.c@{CRC\_\-Test.c}!stripCRC@{stripCRC}}
\index{stripCRC@{stripCRC}!CRC_Test.c@{CRC\_\-Test.c}}
\subsubsection[stripCRC]{\setlength{\rightskip}{0pt plus 5cm}\hyperlink{define_8h_a29}{BOOL} strip\-CRC (\hyperlink{struct_f_r_a_m_e_r}{FRAMER} $\ast$ {\em p\-Framer})}}
\label{_c_r_c___test_8c_a1}


check and strip the CRC16 of the frame. \begin{Desc}
\item[Parameters:]
\begin{description}
\item[{\em p\-Framer}]A frame containing the received data. \end{description}
\end{Desc}
\begin{Desc}
\item[Returns:]TRUE when correct CRC, else return FALSE, when the CRC is correct, then the CRC will be removed from the frame. \end{Desc}
