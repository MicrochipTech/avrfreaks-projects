\hypertarget{timer_8c}{
\section{timer.c File Reference}
\label{timer_8c}\index{timer.c@{timer.c}}
}
{\tt \#include \char`\"{}define.h\char`\"{}}\par
\subsection*{Functions}
\begin{CompactItemize}
\item 
\hyperlink{timer_8c_a0}{INTERRUPT} (SIG\_\-OUTPUT\_\-COMPARE2)
\item 
void \hyperlink{timer_8c_a1}{init\-Timer2} (void)
\item 
void \hyperlink{timer_8c_a2}{stop\-Timer2} (void)
\end{CompactItemize}


\subsection{Detailed Description}
Timer functions used for controlling the time-out of the communication: You still have to fill in what to do when the message isn't sent correctly (e.g. send the frame again, generate an error, ...). 

\subsection{Function Documentation}
\hypertarget{timer_8c_a1}{
\index{timer.c@{timer.c}!initTimer2@{initTimer2}}
\index{initTimer2@{initTimer2}!timer.c@{timer.c}}
\subsubsection[initTimer2]{\setlength{\rightskip}{0pt plus 5cm}void init\-Timer2 (void)}}
\label{timer_8c_a1}


Initialize timer2, so that it will generate an interrupt after 5s. \hypertarget{timer_8c_a0}{
\index{timer.c@{timer.c}!INTERRUPT@{INTERRUPT}}
\index{INTERRUPT@{INTERRUPT}!timer.c@{timer.c}}
\subsubsection[INTERRUPT]{\setlength{\rightskip}{0pt plus 5cm}INTERRUPT (SIG\_\-OUTPUT\_\-COMPARE2)}}
\label{timer_8c_a0}


ISR of Timer2, will be called every 5s when timer2 is enabled. \hypertarget{timer_8c_a2}{
\index{timer.c@{timer.c}!stopTimer2@{stopTimer2}}
\index{stopTimer2@{stopTimer2}!timer.c@{timer.c}}
\subsubsection[stopTimer2]{\setlength{\rightskip}{0pt plus 5cm}void stop\-Timer2 (void)}}
\label{timer_8c_a2}


Stop timer2 by disconnecting clock source and disabling interrupts on this timer. 