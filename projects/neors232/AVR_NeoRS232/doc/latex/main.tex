This project implements an Idle RQ Stop \& Wait protocol. This is a complicated manner to say that a transmitter must wait for an answer of the receiver before it can send something again. The receiver checks the incoming frame using the CRC16-checksum. If the checksum is correct, then an acknowledge is sent back. In the meanwhile, the sender had started a timer after sending his frame. It waits for the acknowledge of the receiver. If that acknowledge doesn't arrive in five seconds, then it tries to resend the data. \par
 A frame consists of a startflag (0x\-AA), the data and a stop flag (0x\-FF). If a 0x\-FF (or some other flag character) occurs in the data, the receiver would falsely think that the end of the frame has been reached. To prevent this, bytestuffing is used. Before every flag character in the data we send the DLE-flag. \par
 When the receiver sees the DLE-flag, it knows that the next character must be considered as a databyte and not as a flag byte.

\par
 Oh, remember: to use this package, you must have some version of javacomm installed. It's available from the \href{http://java.sun.com}{\tt Java} website.

 \begin{Desc}
\item[Author:]Lie\-Btrau 

\href{mailto:locomotieff@tiscali.be}{\tt locomotieff@tiscali.be} \end{Desc}
\begin{Desc}
\item[Version:]version 1.1 \end{Desc}
