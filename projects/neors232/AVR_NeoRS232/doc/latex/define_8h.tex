\hypertarget{define_8h}{
\section{define.h File Reference}
\label{define_8h}\index{define.h@{define.h}}
}
{\tt \#include $<$avr/io.h$>$}\par
{\tt \#include $<$avr/crc16.h$>$}\par
{\tt \#include $<$avr/signal.h$>$}\par
{\tt \#include $<$avr/interrupt.h$>$}\par
\subsection*{Defines}
\begin{CompactItemize}
\item 
\#define \hyperlink{define_8h_a0}{\_\-DEFINE\_\-H\_\-}~1
\item 
\#define \hyperlink{define_8h_a1}{FOSC}~16000000
\item 
\hypertarget{define_8h_a2}{
\#define \hyperlink{define_8h_a2}{FRAMESIZE}~100}
\label{define_8h_a2}

\begin{CompactList}\small\item\em USART receive buffer size. \item\end{CompactList}\end{CompactItemize}
\subsection*{Enumerations}
\begin{CompactItemize}
\item 
enum \hyperlink{define_8h_a27}{FLAGS} \{ \par
\hyperlink{define_8h_a27a3}{LEADING\_\-FLAG} = 0x\-AA, 
\par
\hyperlink{define_8h_a27a4}{DLE\_\-FLAG} = 0x10, 
\par
\hyperlink{define_8h_a27a5}{TRAILING\_\-FLAG} = 0x\-FF, 
\par
\hyperlink{define_8h_a27a6}{ACK0\_\-FLAG} = 0x30
 \}
\begin{CompactList}\small\item\em USART protocol constants. \item\end{CompactList}\item 
enum \hyperlink{define_8h_a28}{FRAME\_\-TYPE} \{ \par
\hyperlink{define_8h_a28a7}{INFO\_\-FRAME} = 0, 
\par
\hyperlink{define_8h_a28a8}{ACK\_\-FRAME} = 1
 \}
\begin{CompactList}\small\item\em The types of frames that can occur in the protocol. \item\end{CompactList}\item 
enum \hyperlink{define_8h_a29}{BOOL} \{ \par
\hyperlink{define_8h_a29a9}{FALSE}, 
\par
\hyperlink{define_8h_a29a10}{TRUE}
 \}
\begin{CompactList}\small\item\em Implementation of boolean type. \item\end{CompactList}\item 
enum \hyperlink{define_8h_a30}{MSG\_\-T} \{ \par
\hyperlink{define_8h_a30a11}{MSG\_\-NO\_\-INPUT}, 
\par
\hyperlink{define_8h_a30a12}{MSG\_\-DECODING}, 
\par
\hyperlink{define_8h_a30a13}{MSG\_\-RX\_\-COMPLETE}
 \}
\begin{CompactList}\small\item\em States of the receiving state machine. \item\end{CompactList}\end{CompactItemize}
\subsection*{Functions}
\begin{CompactItemize}
\item 
void \hyperlink{define_8h_a14}{init\-Timer2} (void)
\item 
void \hyperlink{define_8h_a15}{stop\-Timer2} (void)
\item 
void \hyperlink{define_8h_a16}{write\-String} (uint8\_\-t $\ast$buffer, uint8\_\-t length, \hyperlink{define_8h_a28}{FRAME\_\-TYPE} type\-Frame)
\item 
\hyperlink{define_8h_a30}{MSG\_\-T} \hyperlink{define_8h_a17}{Framer\-Receive} (\hyperlink{struct_f_r_a_m_e_r}{FRAMER} $\ast$p\-Fr)
\item 
void \hyperlink{define_8h_a18}{write\-Frame} (uint8\_\-t $\ast$buffer, uint8\_\-t length)
\item 
\hyperlink{define_8h_a29}{BOOL} \hyperlink{define_8h_a19}{strip\-CRC} (\hyperlink{struct_f_r_a_m_e_r}{FRAMER} $\ast$p\-Framer)
\item 
void \hyperlink{define_8h_a20}{append\-CRC} (uint8\_\-t $\ast$buffer, int length)
\item 
void \hyperlink{define_8h_a21}{init\_\-USART} (uint32\_\-t baud, \hyperlink{define_8h_a29}{BOOL} RX\_\-ON)
\item 
void \hyperlink{define_8h_a22}{write\_\-USART} (uint8\_\-t $\ast$buf, uint8\_\-t length)
\item 
void \hyperlink{define_8h_a23}{write\_\-USART\_\-B} (uint8\_\-t c)
\item 
\hyperlink{define_8h_a29}{BOOL} \hyperlink{define_8h_a24}{read\_\-USART} (uint8\_\-t $\ast$c)
\item 
void \hyperlink{define_8h_a25}{receive\-Init} (\hyperlink{struct_f_r_a_m_e_r}{FRAMER} $\ast$p\-Fr)
\item 
void \hyperlink{define_8h_a26}{receive\-Run} (\hyperlink{struct_f_r_a_m_e_r}{FRAMER} $\ast$p\-Framer)
\end{CompactItemize}


\subsection{Detailed Description}


\subsection{Define Documentation}
\hypertarget{define_8h_a0}{
\index{define.h@{define.h}!_DEFINE_H_@{\_\-DEFINE\_\-H\_\-}}
\index{_DEFINE_H_@{\_\-DEFINE\_\-H\_\-}!define.h@{define.h}}
\subsubsection[\_\-DEFINE\_\-H\_\-]{\setlength{\rightskip}{0pt plus 5cm}\#define \_\-DEFINE\_\-H\_\-~1}}
\label{define_8h_a0}


$<$Don't include this file twice \hypertarget{define_8h_a1}{
\index{define.h@{define.h}!FOSC@{FOSC}}
\index{FOSC@{FOSC}!define.h@{define.h}}
\subsubsection[FOSC]{\setlength{\rightskip}{0pt plus 5cm}\#define FOSC~16000000}}
\label{define_8h_a1}


MCU operating frequency 

\subsection{Enumeration Type Documentation}
\hypertarget{define_8h_a29}{
\index{define.h@{define.h}!BOOL@{BOOL}}
\index{BOOL@{BOOL}!define.h@{define.h}}
\subsubsection[BOOL]{\setlength{\rightskip}{0pt plus 5cm}enum \hyperlink{define_8h_a29}{BOOL}}}
\label{define_8h_a29}


Implementation of boolean type. 

\begin{Desc}
\item[Enumeration values: ]\par
\begin{description}
\index{FALSE@{FALSE}!define.h@{define.h}}\index{define.h@{define.h}!FALSE@{FALSE}}\item[{\em 
\hypertarget{define_8h_a29a9}{
FALSE}
\label{define_8h_a29a9}
}]False, 0 \index{TRUE@{TRUE}!define.h@{define.h}}\index{define.h@{define.h}!TRUE@{TRUE}}\item[{\em 
\hypertarget{define_8h_a29a10}{
TRUE}
\label{define_8h_a29a10}
}]True, 1 \end{description}
\end{Desc}

\hypertarget{define_8h_a27}{
\index{define.h@{define.h}!FLAGS@{FLAGS}}
\index{FLAGS@{FLAGS}!define.h@{define.h}}
\subsubsection[FLAGS]{\setlength{\rightskip}{0pt plus 5cm}enum \hyperlink{define_8h_a27}{FLAGS}}}
\label{define_8h_a27}


USART protocol constants. 

\begin{Desc}
\item[Enumeration values: ]\par
\begin{description}
\index{LEADING_FLAG@{LEADING\_\-FLAG}!define.h@{define.h}}\index{define.h@{define.h}!LEADING_FLAG@{LEADING\_\-FLAG}}\item[{\em 
\hypertarget{define_8h_a27a3}{
LEADING\_\-FLAG}
\label{define_8h_a27a3}
}]Start flag \index{DLE_FLAG@{DLE\_\-FLAG}!define.h@{define.h}}\index{define.h@{define.h}!DLE_FLAG@{DLE\_\-FLAG}}\item[{\em 
\hypertarget{define_8h_a27a4}{
DLE\_\-FLAG}
\label{define_8h_a27a4}
}]Bytestuffing flag \index{TRAILING_FLAG@{TRAILING\_\-FLAG}!define.h@{define.h}}\index{define.h@{define.h}!TRAILING_FLAG@{TRAILING\_\-FLAG}}\item[{\em 
\hypertarget{define_8h_a27a5}{
TRAILING\_\-FLAG}
\label{define_8h_a27a5}
}]End flag \index{ACK0_FLAG@{ACK0\_\-FLAG}!define.h@{define.h}}\index{define.h@{define.h}!ACK0_FLAG@{ACK0\_\-FLAG}}\item[{\em 
\hypertarget{define_8h_a27a6}{
ACK0\_\-FLAG}
\label{define_8h_a27a6}
}]Acknowledge flag \end{description}
\end{Desc}

\hypertarget{define_8h_a28}{
\index{define.h@{define.h}!FRAME_TYPE@{FRAME\_\-TYPE}}
\index{FRAME_TYPE@{FRAME\_\-TYPE}!define.h@{define.h}}
\subsubsection[FRAME\_\-TYPE]{\setlength{\rightskip}{0pt plus 5cm}enum \hyperlink{define_8h_a28}{FRAME\_\-TYPE}}}
\label{define_8h_a28}


The types of frames that can occur in the protocol. 

\begin{Desc}
\item[Enumeration values: ]\par
\begin{description}
\index{INFO_FRAME@{INFO\_\-FRAME}!define.h@{define.h}}\index{define.h@{define.h}!INFO_FRAME@{INFO\_\-FRAME}}\item[{\em 
\hypertarget{define_8h_a28a7}{
INFO\_\-FRAME}
\label{define_8h_a28a7}
}]Info frame (contains data) \index{ACK_FRAME@{ACK\_\-FRAME}!define.h@{define.h}}\index{define.h@{define.h}!ACK_FRAME@{ACK\_\-FRAME}}\item[{\em 
\hypertarget{define_8h_a28a8}{
ACK\_\-FRAME}
\label{define_8h_a28a8}
}]Acknowledge frame \end{description}
\end{Desc}

\hypertarget{define_8h_a30}{
\index{define.h@{define.h}!MSG_T@{MSG\_\-T}}
\index{MSG_T@{MSG\_\-T}!define.h@{define.h}}
\subsubsection[MSG\_\-T]{\setlength{\rightskip}{0pt plus 5cm}enum \hyperlink{define_8h_a30}{MSG\_\-T}}}
\label{define_8h_a30}


States of the receiving state machine. 

\begin{Desc}
\item[Enumeration values: ]\par
\begin{description}
\index{MSG_NO_INPUT@{MSG\_\-NO\_\-INPUT}!define.h@{define.h}}\index{define.h@{define.h}!MSG_NO_INPUT@{MSG\_\-NO\_\-INPUT}}\item[{\em 
\hypertarget{define_8h_a30a11}{
MSG\_\-NO\_\-INPUT}
\label{define_8h_a30a11}
}]No data has come in. \index{MSG_DECODING@{MSG\_\-DECODING}!define.h@{define.h}}\index{define.h@{define.h}!MSG_DECODING@{MSG\_\-DECODING}}\item[{\em 
\hypertarget{define_8h_a30a12}{
MSG\_\-DECODING}
\label{define_8h_a30a12}
}]State machine is decoding data. \index{MSG_RX_COMPLETE@{MSG\_\-RX\_\-COMPLETE}!define.h@{define.h}}\index{define.h@{define.h}!MSG_RX_COMPLETE@{MSG\_\-RX\_\-COMPLETE}}\item[{\em 
\hypertarget{define_8h_a30a13}{
MSG\_\-RX\_\-COMPLETE}
\label{define_8h_a30a13}
}]Complete packet received. \end{description}
\end{Desc}



\subsection{Function Documentation}
\hypertarget{define_8h_a20}{
\index{define.h@{define.h}!appendCRC@{appendCRC}}
\index{appendCRC@{appendCRC}!define.h@{define.h}}
\subsubsection[appendCRC]{\setlength{\rightskip}{0pt plus 5cm}void append\-CRC (uint8\_\-t $\ast$ {\em buffer}, int {\em length})}}
\label{define_8h_a20}


Calculate a CRC16 of a buffer and append it to that buffer. \begin{Desc}
\item[Parameters:]
\begin{description}
\item[{\em buffer}]databuffer of which CRC16 will be calculated and to which CRC16 will be appended. \item[{\em length}]number of bytes in the buffer \end{description}
\end{Desc}
\begin{Desc}
\item[Returns:]the same buffer, with CRC-appended (little endian) \end{Desc}
\hypertarget{define_8h_a17}{
\index{define.h@{define.h}!FramerReceive@{FramerReceive}}
\index{FramerReceive@{FramerReceive}!define.h@{define.h}}
\subsubsection[FramerReceive]{\setlength{\rightskip}{0pt plus 5cm}\hyperlink{define_8h_a30}{MSG\_\-T} Framer\-Receive (\hyperlink{struct_f_r_a_m_e_r}{FRAMER} $\ast$ {\em p\-Fr})}}
\label{define_8h_a17}


Function that reads the received characters from the circular buffer using a state machine. The databytes are read from the circular buffer, a check if performed if it is a flagbyte. If it is, then the correct action is undertaken. If it is a databyte then that databyte is added to the framebuffer. A value is returned that tells us what this function has done with the data (if there was any). \begin{Desc}
\item[Parameters:]
\begin{description}
\item[{\em p\-Fr}]Framer to which the data will be posted. \end{description}
\end{Desc}
\begin{Desc}
\item[Returns:]state of the state machine. \end{Desc}
\hypertarget{define_8h_a21}{
\index{define.h@{define.h}!init_USART@{init\_\-USART}}
\index{init_USART@{init\_\-USART}!define.h@{define.h}}
\subsubsection[init\_\-USART]{\setlength{\rightskip}{0pt plus 5cm}void init\_\-USART (uint32\_\-t {\em baud}, \hyperlink{define_8h_a29}{BOOL} {\em RX\_\-ON})}}
\label{define_8h_a21}


Set the USART with baud, enable RX and TX, enable IRQ on receiving data. \begin{Desc}
\item[Parameters:]
\begin{description}
\item[{\em baud}]Baudrate in \mbox{[}bps\mbox{]} \item[{\em RX\_\-ON}]Enable receiving UART-data when true. \end{description}
\end{Desc}
\hypertarget{define_8h_a14}{
\index{define.h@{define.h}!initTimer2@{initTimer2}}
\index{initTimer2@{initTimer2}!define.h@{define.h}}
\subsubsection[initTimer2]{\setlength{\rightskip}{0pt plus 5cm}void init\-Timer2 (void)}}
\label{define_8h_a14}


Initialize timer2, so that it will generate an interrupt after 5s. \hypertarget{define_8h_a24}{
\index{define.h@{define.h}!read_USART@{read\_\-USART}}
\index{read_USART@{read\_\-USART}!define.h@{define.h}}
\subsubsection[read\_\-USART]{\setlength{\rightskip}{0pt plus 5cm}\hyperlink{define_8h_a29}{BOOL} read\_\-USART (uint8\_\-t $\ast$ {\em c})}}
\label{define_8h_a24}


Function that reads a character from the circular receive buffer. \begin{Desc}
\item[Parameters:]
\begin{description}
\item[{\em c}]Pointer to a character. The read character will be written to that address. \end{description}
\end{Desc}
\hypertarget{define_8h_a25}{
\index{define.h@{define.h}!receiveInit@{receiveInit}}
\index{receiveInit@{receiveInit}!define.h@{define.h}}
\subsubsection[receiveInit]{\setlength{\rightskip}{0pt plus 5cm}void receive\-Init (\hyperlink{struct_f_r_a_m_e_r}{FRAMER} $\ast$ {\em p\-Fr})}}
\label{define_8h_a25}


Initialize the frame so that it is ready to receive data \begin{Desc}
\item[Parameters:]
\begin{description}
\item[{\em p\-Fr}]Pointer to the frame that will be initialized. \end{description}
\end{Desc}
\hypertarget{define_8h_a26}{
\index{define.h@{define.h}!receiveRun@{receiveRun}}
\index{receiveRun@{receiveRun}!define.h@{define.h}}
\subsubsection[receiveRun]{\setlength{\rightskip}{0pt plus 5cm}void receive\-Run (\hyperlink{struct_f_r_a_m_e_r}{FRAMER} $\ast$ {\em p\-Framer})}}
\label{define_8h_a26}


Receive run routine: checks if data has come in and act accordingly. \begin{Desc}
\item[Parameters:]
\begin{description}
\item[{\em p\-Framer}]Check if data has come in on this framer \end{description}
\end{Desc}
\hypertarget{define_8h_a15}{
\index{define.h@{define.h}!stopTimer2@{stopTimer2}}
\index{stopTimer2@{stopTimer2}!define.h@{define.h}}
\subsubsection[stopTimer2]{\setlength{\rightskip}{0pt plus 5cm}void stop\-Timer2 (void)}}
\label{define_8h_a15}


Stop timer2 by disconnecting clock source and disabling interrupts on this timer. \hypertarget{define_8h_a19}{
\index{define.h@{define.h}!stripCRC@{stripCRC}}
\index{stripCRC@{stripCRC}!define.h@{define.h}}
\subsubsection[stripCRC]{\setlength{\rightskip}{0pt plus 5cm}\hyperlink{define_8h_a29}{BOOL} strip\-CRC (\hyperlink{struct_f_r_a_m_e_r}{FRAMER} $\ast$ {\em p\-Framer})}}
\label{define_8h_a19}


check and strip the CRC16 of the frame. \begin{Desc}
\item[Parameters:]
\begin{description}
\item[{\em p\-Framer}]A frame containing the received data. \end{description}
\end{Desc}
\begin{Desc}
\item[Returns:]TRUE when correct CRC, else return FALSE, when the CRC is correct, then the CRC will be removed from the frame. \end{Desc}
\hypertarget{define_8h_a22}{
\index{define.h@{define.h}!write_USART@{write\_\-USART}}
\index{write_USART@{write\_\-USART}!define.h@{define.h}}
\subsubsection[write\_\-USART]{\setlength{\rightskip}{0pt plus 5cm}void write\_\-USART (uint8\_\-t $\ast$ {\em buf}, uint8\_\-t {\em length})}}
\label{define_8h_a22}


Send a characterbuffer to the USART \begin{Desc}
\item[Parameters:]
\begin{description}
\item[{\em buf}]Pointer to buffer that contains data to be sent \item[{\em length}]Number of bytes to be sent \end{description}
\end{Desc}
\hypertarget{define_8h_a23}{
\index{define.h@{define.h}!write_USART_B@{write\_\-USART\_\-B}}
\index{write_USART_B@{write\_\-USART\_\-B}!define.h@{define.h}}
\subsubsection[write\_\-USART\_\-B]{\setlength{\rightskip}{0pt plus 5cm}void write\_\-USART\_\-B (uint8\_\-t {\em c})}}
\label{define_8h_a23}


Write one character to the UART \begin{Desc}
\item[Parameters:]
\begin{description}
\item[{\em c}]The character that must be sent. \end{description}
\end{Desc}
\hypertarget{define_8h_a18}{
\index{define.h@{define.h}!writeFrame@{writeFrame}}
\index{writeFrame@{writeFrame}!define.h@{define.h}}
\subsubsection[writeFrame]{\setlength{\rightskip}{0pt plus 5cm}void write\-Frame (uint8\_\-t $\ast$ {\em buffer}, uint8\_\-t {\em length})}}
\label{define_8h_a18}


Function that sends a frame. First a CRC16 is appended. Then the frame is sent. After that, a timeout timer is initialized. The acknowledge must come in before the timeout expires. \begin{Desc}
\item[Parameters:]
\begin{description}
\item[{\em buffer}]Buffer containing the data to send. \item[{\em length}]Number of bytes that must be sent. \end{description}
\end{Desc}
\hypertarget{define_8h_a16}{
\index{define.h@{define.h}!writeString@{writeString}}
\index{writeString@{writeString}!define.h@{define.h}}
\subsubsection[writeString]{\setlength{\rightskip}{0pt plus 5cm}void write\-String (uint8\_\-t $\ast$ {\em buffer}, uint8\_\-t {\em length}, \hyperlink{define_8h_a28}{FRAME\_\-TYPE} {\em type\-Frame})}}
\label{define_8h_a16}


Write a databuffer to the serial port. No CRC will be appended. This function adds the flags for the protocol and performs the bytestuffing. Finally it sends the data to the UART. \begin{Desc}
\item[Parameters:]
\begin{description}
\item[{\em buffer}]Buffer containing the data. \item[{\em length}]Length of the databuffer \item[{\em type\-Frame}]Type of frame (info frame or acknowledge frame) \end{description}
\end{Desc}
